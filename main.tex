\documentclass[12pt,italian]{report}
\usepackage{tesi}

\def\myCDL{Corso di Laurea magistrale in\\Informatica}

% TITOLO TESI:
\def\myTitle{Studio e Sviluppo di un Sistema di Passthrough per TEE tra QEMU e Linux su Piattaforme ARM}

% AUTORE:
\def\myName{Marco Cutecchia}
\def\myMat{Matr. Nr. 983828}

% RELATORE E CORRELATORE:
\def\myRefereeA{Prof. Danilo Bruschi}

% ANNO ACCADEMICO
\def\myYY{2022-2023}

% Il seguente comando introduce un elenco delle figure dopo l'indice (facoltativo)
%\figurespagetrue

% Il seguente comando introduce un elenco delle tabelle dopo l'indice (facoltativo)
%\tablespagetrue

\usepackage[a4paper]{geometry}		% Formato del foglio
\usepackage[italian]{babel}			% Supporto per l'italiano
\usepackage[utf8]{inputenc}			% Supporto per UTF-8
\usepackage[a-1b]{pdfx}				% File conforme allo standard PDF-A (obbligatorio per la consegna)

\usepackage{graphicx}				% Funzioni avanzate per le immagini
\usepackage{hologo}					% Bibtex logo with \hologo{BibTeX}
%\usepackage{epsfig}				% Permette immagini in EPS
%\usepackage{xcolor}				% Gestione avanzata dei colori
\usepackage{amssymb,amsmath,amsthm} % Simboli matematici
\usepackage{listings}				% Scrittura di codice
\usepackage{url}					% Visualizza e rendere interattii gli URL
\usepackage{hyperref}				% Rende interattivi i collegamenti interni


\begin{document}

\frontespizio
\afterpreface

\chapter{Introduzione}
\label{cap:introduzione}
% + Breve spiegazione di cosa è un TEE
% + Molto diffusi su mobile, inizia ad esserci interesse su piattaforme cloud (trusted computing)
% + Esistono soluzioni ma sono solo su piattaforme x86 via intel SGX che è una implementazione molto particolare(approfondire)
% + Con il crescente interesse nell'uso di ARM sui server vogliamo scoprire come integrare un TEE implementato via TrustZone in ambienti virtualizzati
% + Implementato un prototipo tramite QEMU, Linux ed utilizzando OP-TEE come test case di TEE
% + Piccolo risultato di "assaggio" 

\chapter{Trusted Execution Environment}
\label{sec:tee}

% Descrizione dettagliata di cosa è un TEE
	% + Ambiente in cui è garantita l'integrita del codice in esecuzione
	% + Root of Trust
	% + Isolazione rispetto al kernel, superficie di attacco minore
	% + Livello di privilegio superiore al kernel, può nascondere la presenza di device ad esso

\section{Applicazioni dei TEE}
\label{sec:applicazioni-tee}
% Oggi: DRM, Secure Storage, Secure Boot, Crypto Accelerator
% Domani: Remote Attestation, ???

\section{TEE Disponibili}
\label{sec:tee-disponibili}
% Parli di OP-TEE, AMD PSP, Apple Secure Enclave, Samsung Knox, Trusty...
% Parli della segretezza delle info disponibili

\section{Trusted Computing Initiative}
\label{sec:trusted-computing-initiative}
% Obiettivo: permettere ai clienti di piattaforme cloud di potersi fidare dei risultati
% TEE è un componente fondamentale
% Al momento le soluzioni disponibili sono solo su x86 con Intel SGX
% Descrizione di come funzionano in questo caso

\section{Questioni Etiche nell'Uso dei TEE}
\label{sec:etica-tee}
% Una root of trust intagliata nell'hardware significa che non abbiamo il pieno controllo sul SW in esecuzione sulla nostra macchina
% La maggior parte dei vendor tiene un velo di segretezza rispetto ai propri TEE
% Un TEE può avere controllo su tutto il computer senza che l'OS possa notarlo
% Un malintenzionato che prende controllo di un TEE può insediarsi in un computer in modo da essere impossibile da intercettare
% Sarebbe interessante studiare modi di garantire la root of trust mantenendo l'inspectability (codice Open source prob non basta)

\chapter{Passthrough per TEE tra QEMU e Linux}
\label{chap:passthrough-tee-qemu-linux}
ciao test \cite{lim2019web}.
% Breve intro a QEMU
% Descrizione di quello che vogliamo fare (permettere ad un TEE generico di offrire servizi a macchine virtuali)
% Interessante sia in ambito "casalingo" sia in ambito cloud per i provider

\section{ARM TrustZone}
\label{sec:arm-trustzone}
% Cosa è
% Come funziona
% Non c'è supporto alla virtualizzazione prima di ARM v8-B (che non è ancora uscito)

\section{GlobalPlatform API}
\label{sec:global-platform-api}
% Cosa è, cosa chiede di implementare
% (?) Forse poco per una sezione?

\section{Il Sottosistema TEE per Linux}
\label{sec:sottosistema-tee-per-linux}
% Cosa è, perchè lo usiamo (perchè è una interfaccia comune)
% Funzionamento via ioctl

\section{Implementazione tramite l'Estensione del Kernel}
\label{sec:implementazione-passthrough-tramite-kernel-module}
% Descrizione del modulo Linux, modifiche a QEMU
% Comunicazione via MMIO
% Conversione di indirizzi virt/phys
% Gestione della shared memory
% Pro: è esplicito l'uso di un passthrough, facilmente cambiabile il passthrough sopra
% Contro: necessario estendere il kernel con un modulo

\section{Implementazione tramite l'Emulazione di OP-TEE}
\label{sec:implementazione-passthrough-tramite-bootloader}
% Cosa è OP-TEE
% Bootloader custom
% Descrizione del passaggio di parametri tra OP-TEE e bootloader
% Pro: passthrough invisibile, non serve estendere il kernel
% Contro: legato strettamente ad una singola versione di OP-TEE

\section{Verifica Funzionale tramite Test Suite di OP-TEE}
\label{sec:verifica-funzionale-via-optee}
% Utilizzo della test suite di OP TEE per verificare la correttezza funzionale

\chapter{Conclusioni}
\label{chap:conclusioni}


\section{Sviluppi Futuri}
\label{sec:sviluppi-futuri}

\appendix
\chapter{FakeTEE: Simulatore di TEE per aiutare nello sviluppo}
% + Durante lo sviluppo era necessario avere un TEE ispezionabile facilmente
% + Con QEMU è possibile emularlo ma dato che volevo avere un TEE sull'host
%	accessibile da una VM guest sarebbe stato necessario avere una VM dentro
%	la VM durante lo sviluppo (scomodo!)
% + FakeTEE è un piccolo modulo per il kernel linux che si attacca al TEE subsystem
%	e finge la presenza di un TEE
% + Fa tutto in kernel space, non mantiene stato ed è facilmente modificabile e reloadable
% + Utile durante lo sviluppo di sw legato ai TEE 

\bibliographystyle{unsrt}
\bibliography{bibliografia}
\addcontentsline{toc}{chapter}{Bibliografia}

\end{document}
